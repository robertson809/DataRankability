% document class and packages
\documentclass{beamer}
\usepackage{algorithm,algorithmic}
\usepackage{amsmath}
\usepackage{amssymb}
\usepackage{color, colortbl}
\usepackage{graphicx}
\usepackage{hyperref}
\usepackage{pgfplots}
\usepackage{tikz}

\newlength\myindent
\setlength\myindent{1em}
\newcommand\bindent{%
  \begingroup
  \setlength{\itemindent}{\myindent}
  \addtolength{\algorithmicindent}{\myindent}
}
\newcommand\eindent{\endgroup}

% new commands and operators
\DeclareMathOperator{\md}{md}
\DeclareMathOperator{\sv}{sv}
\DeclareMathOperator{\hd}{hd}
\newcommand\abs[1]{\left|#1\right|}
\newcommand\norm[1]{\left\Vert#1\right\Vert}
\newcommand{\iu}{{i\mkern1mu}}

\newtheorem{proposition}[theorem]{Proposition}

% remove figure caption prefix
\setbeamertemplate{caption}{\raggedright\insertcaption\par}

% hyperlinks setup
\hypersetup{colorlinks,breaklinks,
	urlcolor=[rgb]{0,0.75,1},
	linkcolor=[rgb]{0.75,0.75,0.75}}

% empty navigation symbols
\beamertemplatenavigationsymbolsempty

% remove navigation dots on miniframes
\makeatletter
\def\beamer@writeslidentry{\clearpage\beamer@notesactions}
\makeatother

% Use Theme
\usetheme{Warsaw}
\useoutertheme[footline=authortitle]{miniframes}
\useinnertheme[shadow=true]{rounded}

% Colors
\definecolor{primary}{RGB}{0, 0, 0} % (primary, black)
\definecolor{secondary}{RGB}{102, 178, 255} % (secondary, light blue)
\definecolor{text}{RGB}{190,190,190} % (text, light silver)

% Beamer Colors
\setbeamercolor{palette primary}{bg=primary,fg=text}
\setbeamercolor{palette secondary}{bg=secondary,fg=text}
\setbeamercolor{palette tertiary}{bg=primary,fg=text}
\setbeamercolor{structure}{fg=primary} % itemize, enumerate, etc
\setbeamercolor{frametitle}{fg=primary}

% Transparency for itemized listing
%\setbeamercovered{transparent}

% Title Page
\title{On the Algebraic Connectivity of Directed Graphs}
\author{Thomas R. Cameron}
\institute{Davidson College}
\date{May 7, 2019}

\begin{document}
% Title Frame
\begin{frame}
	\titlepage
\end{frame}

% Outline
%\AtBeginSection[]{
 %\frame<beamer>{
  %\frametitle{Outline}   
  %\tableofcontents[currentsection]
 %}
%}

%%%%%%%%%%%%%%%%%%%%%%%%%%%%%%%%%%%%%%%%%%%%%%%%%%%%%%
%								Algebraic Connectivity
%%%%%%%%%%%%%%%%%%%%%%%%%%%%%%%%%%%%%%%%%%%%%%%%%%%%%%
\section{Algebraic Connectivity}

\begin{frame}{Definition}
In~\cite{Wu2005-1}, the algebraic connectivity of a directed graph $G$ on $n$ vertices is defined by~\footnote{Another definition of algebraic connectivity is given in~\cite{Wu2005-2}}.
\vfill
\[
\alpha(G) = \min_{x\in P}x^{T}Lx,
\]
\vfill
where $L$ is the graph Laplacian and
\vfill
\[
P= \{x\in\mathbb{R}^{n}\colon~\norm{x}=1,~x\perp e\}
\]
\end{frame}

\begin{frame}{Properties}
\begin{itemize}
\item	$\alpha(G)$ is independent of the ordering of vertices since $P$ is an invariant subspace of permutation matrices.
\vfill
\item	Let $Q$ be a unitary matrix whose columns span $P$, then
	\[
	\alpha(G) = \lambda_{\text{min}}\left(\frac{1}{2}Q^{T}\left(L+L^{T}\right)Q\right)
	\]
\end{itemize}
\end{frame}

\begin{frame}{Properties}
\begin{itemize}
\item	If the graph $G$ is undirected, then the definition of $\alpha(G)$ coincides with Fiedler's definition of algebraic connectivity~\cite{Fiedler1973}.
\vfill
\item For a graph $G$ with Laplacian $L$, we have
	\[
	\lambda_{1}\left(\frac{1}{2}\left(L+L^{T}\right)\right)\leq\alpha(G)\leq\lambda_{2}\left(\frac{1}{2}\left(L+L^{T}\right)\right).
	\]	
\end{itemize}
\end{frame}

\begin{frame}{Examples}
\begin{minipage}{0.45\textwidth}
\begin{tikzpicture}
	\node[circle, shading=ball, ball color=gray, color=white] (1) at (-1,2) {$1$};
	\node[circle, shading=ball, ball color=gray, color=white] (2) at (-1,0) {$2$};
	\node[circle, shading=ball, ball color=gray, color=white] (3) at (-1,-2) {$3$};
	\node[circle, shading=ball, ball color=gray, color=white] (4) at (1,2) {$4$};
	\node[circle, shading=ball, ball color=gray, color=white] (5) at (1,0) {$5$};
	\node[circle, shading=ball, ball color=gray, color=white] (6) at (1,-2) {$6$};
	
	\draw[gray,->,thick](1) to [out=270,in=90,looseness=0](2);
	\draw[gray,->,thick](2) to [out=270,in=90,looseness=0](3);
	\draw[gray,->,thick](1) to [out=225,in=135,looseness=1](3);
	\draw[gray,->,thick](4) to [out=270,in=90,looseness=0](5);
	\draw[gray,->,thick](5) to [out=270,in=90,looseness=0](6);
	\draw[gray,->,thick](4) to [out=315,in=45,looseness=1](6);
\end{tikzpicture}
\end{minipage}\hfill
\begin{minipage}{0.45\textwidth}
Eigenvalues of the Laplacian:
\[
\sigma(L)=\{2,1,0,2,1,0\}
\]
\vfill
Spectral Rankability:
\[
\text{rankH}(G) = 0.6
\]
\vfill
Algebraic Connectivity:
\[
\alpha(G) = -0.389
\]
\vfill
Connectivity Rankability:
\begin{align*}
\text{rankA}(G) &= 1.226 \\
&\sim 1.0
\end{align*}
\end{minipage}
\end{frame}

\begin{frame}{Examples}
\begin{minipage}{0.45\textwidth}
\begin{tikzpicture}
	\node[circle, shading=ball, ball color=gray, color=white] (1) at (-1,2) {$1$};
	\node[circle, shading=ball, ball color=gray, color=white] (2) at (-1,0) {$2$};
	\node[circle, shading=ball, ball color=gray, color=white] (3) at (-1,-2) {$3$};
	\node[circle, shading=ball, ball color=gray, color=white] (4) at (1,2) {$4$};
	\node[circle, shading=ball, ball color=gray, color=white] (5) at (1,0) {$5$};
	\node[circle, shading=ball, ball color=gray, color=white] (6) at (1,-2) {$6$};
	
	\draw[gray,->,thick](1) to [out=270,in=90,looseness=0](2);
	\draw[gray,->,thick](2) to [out=270,in=90,looseness=0](3);
	\draw[gray,->,thick](1) to [out=225,in=135,looseness=1](3);
	\draw[gray,->,thick](2) to [out=45,in=225,looseness=0](4);
	\draw[gray,->,thick](4) to [out=270,in=90,looseness=0](5);
	\draw[gray,->,thick](5) to [out=270,in=90,looseness=0](6);
	\draw[gray,->,thick](4) to [out=315,in=45,looseness=1](6);
\end{tikzpicture}
\end{minipage}\hfill
\begin{minipage}{0.45\textwidth}
Eigenvalues of the Laplacian:
\[
\sigma(L)=\{2,2,0,2,1,0\}
\]
\vfill
Spectral Rankability:
\[
\text{rankH}(G) = 0.6
\]
\vfill
Algebraic Connectivity:
\[
\alpha(G) = -0.303
\]
\vfill
Connectivity Rankability:
\begin{align*}
\text{rankA}(G) &= 1.139 \\
&\sim 0.929
\end{align*}
\end{minipage}
\end{frame}

\begin{frame}{Examples}
\begin{minipage}{0.45\textwidth}
\begin{tikzpicture}
	\node[circle, shading=ball, ball color=gray, color=white] (1) at (-1,2) {$1$};
	\node[circle, shading=ball, ball color=gray, color=white] (2) at (-1,0) {$2$};
	\node[circle, shading=ball, ball color=gray, color=white] (3) at (-1,-2) {$3$};
	\node[circle, shading=ball, ball color=gray, color=white] (4) at (1,2) {$4$};
	\node[circle, shading=ball, ball color=gray, color=white] (5) at (1,0) {$5$};
	\node[circle, shading=ball, ball color=gray, color=white] (6) at (1,-2) {$6$};
	
	\draw[gray,->,thick](1) to [out=270,in=90,looseness=0](2);
	\draw[gray,->,thick](2) to [out=270,in=90,looseness=0](3);
	\draw[gray,->,thick](1) to [out=225,in=135,looseness=1](3);
	\draw[gray,->,thick](3) to [out=45,in=225,looseness=0](4);
	\draw[gray,->,thick](4) to [out=270,in=90,looseness=0](5);
	\draw[gray,->,thick](5) to [out=270,in=90,looseness=0](6);
	\draw[gray,->,thick](4) to [out=315,in=45,looseness=1](6);
\end{tikzpicture}
\end{minipage}\hfill
\begin{minipage}{0.45\textwidth}
Eigenvalues of the Laplacian:
\[
\sigma(L)=\{2,1,1,2,1,0\}
\]
\vfill
Spectral Rankability:
\[
\text{rankH}(G) = 0.6
\]
\vfill
Algebraic Connectivity:
\[
\alpha(G) = -0.068
\]
\vfill
Connectivity Rankability:
\begin{align*}
\text{rankA}(G) &= 0.905 \\
&\sim 0.738
\end{align*}
\end{minipage}
\end{frame}

%%%%%%%%%%%%%%%%%%%%%%%%%%%%%%%%%%%%%%%%%%%%%%%%%%%%%%
%								Bibliography
%%%%%%%%%%%%%%%%%%%%%%%%%%%%%%%%%%%%%%%%%%%%%%%%%%%%%%
\begin{frame}[allowframebreaks]{Bibliography}
\bibliographystyle{amsalpha}
\bibliography{../Bibliography}
\end{frame}

\end{document}