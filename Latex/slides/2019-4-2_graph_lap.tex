% document class and packages
\documentclass{beamer}
\usepackage{algorithm,algorithmic}
\usepackage{amsmath}
\usepackage{amssymb}
\usepackage{graphicx}
\usepackage{hyperref}

% new commands and operators
\DeclareMathOperator{\md}{md}
\DeclareMathOperator{\sv}{sv}
\DeclareMathOperator{\hd}{hd}
\newcommand\abs[1]{\left|#1\right|}

\newtheorem{proposition}[theorem]{Proposition}

% remove figure caption prefix
\setbeamertemplate{caption}{\raggedright\insertcaption\par}

% hyperlinks setup
\hypersetup{colorlinks,breaklinks,
	urlcolor=[rgb]{0,0.75,1},
	linkcolor=[rgb]{0.75,0.75,0.75}}

% empty navigation symbols
\beamertemplatenavigationsymbolsempty

% remove navigation dots on miniframes
\makeatletter
\def\beamer@writeslidentry{\clearpage\beamer@notesactions}
\makeatother

% Use Theme
\usetheme{Warsaw}
\useoutertheme[footline=authortitle]{miniframes}
\useinnertheme[shadow=true]{rounded}

% Colors
\definecolor{primary}{RGB}{0, 0, 0} % (primary, black)
\definecolor{secondary}{RGB}{102, 178, 255} % (secondary, light blue)
\definecolor{text}{RGB}{190,190,190} % (text, light silver)

% Beamer Colors
\setbeamercolor{palette primary}{bg=primary,fg=text}
\setbeamercolor{palette secondary}{bg=secondary,fg=text}
\setbeamercolor{palette tertiary}{bg=primary,fg=text}
\setbeamercolor{structure}{fg=primary} % itemize, enumerate, etc
\setbeamercolor{frametitle}{fg=primary}

% Transparency for itemized listing
%\setbeamercovered{transparent}

% Title Page
\title{Graph Laplacian and the Rankability of Data}
\author{Thomas R. Cameron}
\institute{Davidson College}
\date{\today}

\begin{document}
% Title Frame
\begin{frame}
	\titlepage
\end{frame}

% Outline
%\AtBeginSection[]{
 %\frame<beamer>{
  %\frametitle{Outline}   
  %\tableofcontents[currentsection]
 %}
%}

%%%%%%%%%%%%%%%%%%%%%%%%%%%%%%%%%%%%%%%%%%%%%%%%%%%%%%
%								Definitions
%%%%%%%%%%%%%%%%%%%%%%%%%%%%%%%%%%%%%%%%%%%%%%%%%%%%%%
\section{Definitions}

\begin{frame}{Weighted Directed Graphs}
Let $\Gamma=(V,E,w)$ be a weighted directed graph (simple).
\vfill
\begin{itemize}
\item The out degree of vertex $i$ is defined by $d_{i}^{out}=\sum_{j\in V}w_{ij}$.
\vfill
\item	$\Gamma$ is weakly connected if replacing all of its edges with undirected edges produces a connected undirected graph.
\vfill
\item	$\Gamma$ is strongly connected if for any pair of vertices $i\neq j$, there is a path from $i$ to $j$ and a path from $j$ to $i$. 
\end{itemize}
\vfill
\end{frame}

\begin{frame}{Isolated Vertices and Subgraphs}
Let $\Gamma'=(V',E',w')$ be an induced subgraph of $\Gamma$.
\vfill
\begin{itemize}
\item	Vertex $i$ is isolated if $w_{ij}=0$ for all $j\in V$.
\vfill
\item	Vertex $i$ is quasi-isolated if $\sum_{j\in V}w_{ij}=0$.
\vfill
\item	Subgraph $\Gamma'$ is isolated if $w_{ij}=0$ for all $i\in V'$ and $j\notin V'$.
\vfill
\item Subgraph $\Gamma'$ is quasi-isolated if $\sum_{j\in V\setminus{V'}}w_{ij}=0$ for all $i\in V'$.  
\end{itemize}
\vfill
\end{frame}

\begin{frame}{Graph Laplace Operator}
The Laplace operator $\Delta$ of the graph $\Gamma$ is defined by
\vfill
\[
\Delta f(i) = 	\begin{cases}
			f(i)d_{i}^{out} - \sum_{j}w_{ij}f(j) & \text{if $d_{i}^{out}\neq 0$} \\
			0	&	\text{otherwise}
			\end{cases},
\]
\vfill
where $f$ is a complex valued function on $V$.
\vfill
\end{frame}

\begin{frame}{Reduced Graph Laplace Operator}
Let $V_{R}\subseteq V$ be the set of vertices that are not quasi-isolated.
The reduced Laplace operator $\Delta_{R}$ is defined by
\vfill
\[
\Delta_{R}f(i)=f(i)d_{i}^{out} - \sum_{j\in V_{R}}w_{ij}f(j),~i\in V_{R},
\]
\vfill
where $f$ is a complex valued function on $V_{R}$ and $d_{i}^{out}$ is the out degree of vertex $i$ in $\Gamma$.
\vfill
\end{frame}

\begin{frame}{Frobenius Normal Form}
The Frobenius normal form of the Laplace operator:
\vfill
\[
\Delta = 	\left(\begin{array}{cccc}
		\Delta_{1} & \Delta_{12} & \cdots & \Delta_{1z} \\
		0 & \Delta_{2} & \cdots & \Delta_{2z} \\
		\vdots & \vdots & \ddots & \vdots \\
		0 & 0 & \cdots & \Delta_{z}
		\end{array}\right),
\]
\vfill
where $\Delta_{1},\ldots,\Delta_{z}$ are square matrices corresponding to the strongly connected components of $\Gamma$.
\vfill
\end{frame}

%%%%%%%%%%%%%%%%%%%%%%%%%%%%%%%%%%%%%%%%%%%%%%%%%%%%%%
%								Spectral Properties
%%%%%%%%%%%%%%%%%%%%%%%%%%%%%%%%%%%%%%%%%%%%%%%%%%%%%%
\section{Spectral Properties}

\begin{frame}{Basic Properties}
Let $\sigma(\Delta)$ denote the set of eigenvalues of the Laplace operator $\Delta$. 
Then,
\vfill
\begin{itemize}
\item	$0\in\sigma(\Delta)$,
\vfill
\item $\sigma(\Delta)$ is symmetric with respect to the real axis,
\vfill
\item $\sigma(\Delta)=\sigma(\Delta_{R})\cup\{\text{$0$, repeated $\abs{V\setminus{V_{R}}}$ times}\}$,
\vfill
\item	$\sigma(\Delta)=\bigcup_{i}^{z}\sigma(\Delta_{i})$,
\vfill
\item	$\sigma(\Delta)$ is the union of the spectra of the Laplace operator reduced to the weakly connected components of $\Gamma$.
\end{itemize}
\vfill
\end{frame}

\begin{frame}{Spanning Tree}
A graph $\Gamma$ is said to have a spanning tree if there exists a vertex from which all other vertices can be reached following directed edges.
\vfill
\begin{proposition}
Every graph $\Gamma$ contains at least one isolated strongly connected component. 
Furthermore, $\Gamma$ contains exactly one isolated strongly connected component if and only if $\Gamma$ contains a spanning Tree.
\end{proposition}
\end{frame}

\begin{frame}{Isolated Subgraph}
\begin{theorem}
Let $\Gamma$ be a directed graph with non-negative weights, and let $\Gamma_{i}$, $1\leq i\leq z$, be its strongly connected components.
Then, zero is an eigenvalue (in fact a simple eigenvalue) of $\Delta_{i}$ if and only if $\Gamma_{i}$ is isolated. 
\end{theorem}
\vfill
\begin{corollary}
For directed graphs $\Gamma$ with non-negative weights, the following are equivalent:
\begin{enumerate}[i.]
\item	The multiplicity (algebraic) of the zero eigenvalue of $\Delta$ is equal to $k$.
\item	There exist $k$ isolated strongly connected components in $\Gamma$.
\item	The minimum number of directed trees needed to span the whole graph is equal to $k$. 
\end{enumerate}
\end{corollary}
\vfill
\end{frame}

\begin{frame}{Spectral Characterization}
\begin{theorem}
Let $\Gamma$ be a directed graph on $n$-vertices with non-negative weights.
Then, $\sigma(\Delta)=\{d_{1}^{out},\ldots,d_{n}^{out}\}$ if and only if $\Gamma$ is acyclic.
\end{theorem}
\vfill
\begin{corollary}
Let $\Gamma$ be a directed graph on $n$-vertices with binary weights.
Then, $\sigma(\Delta)=\{d_{1}^{out},\ldots,d_{n}^{out}\}$, where $d_{i}^{out}=n-i$ for all $i=1,\ldots,n$, if and only if $\Gamma$ is a perfect dominance graph. 
\end{corollary}
\end{frame}

%%%%%%%%%%%%%%%%%%%%%%%%%%%%%%%%%%%%%%%%%%%%%%%%%%%%%%
%								Rankability Measure
%%%%%%%%%%%%%%%%%%%%%%%%%%%%%%%%%%%%%%%%%%%%%%%%%%%%%%
\section{Rankability Measure}

\begin{frame}{Spectral Distance}
Let $A$ have eigenvalues $\lambda_{1},\ldots,\lambda_{n}$ and $\tilde{A}$ have eigenvalues $\tilde{\lambda}_{1},\ldots,\tilde{\lambda}_{n}$.
\vfill
\begin{itemize}
\item	The spectral variation of $\tilde{A}$ with respect to $A$ is
\[
\sv_{A}(\tilde{A})=\max_{i}\min{j}\abs{\tilde{\lambda}_{i}-\lambda_{j}}.
\]
\vfill
\item	The Hausdorff distance between the eigenvalues of $A$ and $\tilde{A}$ is
\[
\hd(A,\tilde{A})=\max\{sv_{A}(\tilde{A}),\sv_{\tilde{A}}(A)\}
\]
\vfill
\item	The matching distance between the eigenvalues of $A$ and $\tilde{A}$ is
\[
\md(A,\tilde{A})=\min_{\pi}\max_{i}\abs{\tilde{\lambda}_{\pi(i)}-\lambda_{i}}
\]
\end{itemize}
\vfill
\end{frame}

\begin{frame}{Rankability Measure}
\begin{example}
\[
\left(\begin{array}{c | c | c}
Graph & Hausdorff & Matching \\
\hline
Dominance Graph & 0.0 & 0.0 \\
Perturbed Dominance Graph & 0.62 & 0.62 \\
Perturbed Random Graph & 1.23 & 1.70 \\
Nearly Disconnected & 2.0 & 2.0 \\
Random & 2.65 & 2.65 \\
Cyclic & 3.0 & 3.0 \\
Completely Connected & 3.0 & 5.0 \\
Empty Graph & 5.0 & 5.0
\end{array}\right)
\]
\end{example}
\end{frame}


%%%%%%%%%%%%%%%%%%%%%%%%%%%%%%%%%%%%%%%%%%%%%%%%%%%%%%
%								References 
%%%%%%%%%%%%%%%%%%%%%%%%%%%%%%%%%%%%%%%%%%%%%%%%%%%%%%
%\begin{frame}[allowframebreaks]{References}
%\bibliographystyle{amsalpha}
%\bibliography{../Bibliography}
%\end{frame}

\end{document}