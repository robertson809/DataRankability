% document class and packages
\documentclass{beamer}
\usepackage{adjustbox}
\usepackage{algorithm,algorithmic}
\usepackage{amsmath}
\usepackage{amssymb}
\usepackage{color, colortbl}
\usepackage{graphicx}
\usepackage{hyperref}
\usepackage{pgfplots}
\usepackage{tikz}

\newlength\myindent
\setlength\myindent{1em}
\newcommand\bindent{%
  \begingroup
  \setlength{\itemindent}{\myindent}
  \addtolength{\algorithmicindent}{\myindent}
}
\newcommand\eindent{\endgroup}

% new commands and operators
\newcommand\diag[1]{\textrm{diag}\left(#1\right)}
\newcommand\abs[1]{\left|#1\right|}
\newcommand\norm[1]{\left\Vert#1\right\Vert}
\newcommand{\iu}{{i\mkern1mu}}

\newtheorem{proposition}[theorem]{Proposition}

% remove figure caption prefix
\setbeamertemplate{caption}{\raggedright\insertcaption\par}

% hyperlinks setup
\hypersetup{colorlinks,breaklinks,
	urlcolor=[rgb]{0,0.75,1},
	linkcolor=[rgb]{0.75,0.75,0.75}}

% empty navigation symbols
\beamertemplatenavigationsymbolsempty

% remove navigation dots on miniframes
\makeatletter
\def\beamer@writeslidentry{\clearpage\beamer@notesactions}
\makeatother

% Use Theme
\usetheme{Warsaw}
\useoutertheme[footline=authortitle]{miniframes}
\useinnertheme[shadow=true]{rounded}

% Colors
\definecolor{black}{RGB}{0, 0, 0} % (primary, black)
\definecolor{lblue}{RGB}{102, 178, 255} % (secondary, light blue)
\definecolor{lgreen}{RGB}{102, 255, 178} %(tertiary, light green)
\definecolor{lsilver}{RGB}{224,224,224} % (text, light silver)
\definecolor{gray}{RGB}{128,128,128} % (graph node shade, gray)
\definecolor{white}{RGB}{255,255,255} % (graph node text, white)

% Beamer Colors
\setbeamercolor{palette primary}{bg=black,fg=lsilver}
\setbeamercolor{palette secondary}{bg=lblue,fg=lsilver}
\setbeamercolor{palette tertiary}{bg=black,fg=lsilver}
\setbeamercolor{structure}{fg=black} % itemize, enumerate, etc
\setbeamercolor{frametitle}{fg=black}

% Transparency for itemized listing
%\setbeamercovered{transparent}

% Title Page
\title{Introduction to the Graph Laplacian \\ and Rankability}
\author{Thomas R. Cameron}
\institute{Davidson College}
\date{May 17, 2019}

\begin{document}
% Title Frame
\begin{frame}
	\titlepage
\end{frame}

% Outline
%\AtBeginSection[]{
 %\frame<beamer>{
  %\frametitle{Outline}   
  %\tableofcontents[currentsection]
 %}
%}

%%%%%%%%%%%%%%%%%%%%%%%%%%%%%%%%%%%%%%%%%%%%%%%%%%%%%%
%								The Laplacian
%%%%%%%%%%%%%%%%%%%%%%%%%%%%%%%%%%%%%%%%%%%%%%%%%%%%%%
\section{The Laplacian}

\begin{frame}{Weighted Directed Graphs}
We are interested in finite simple loopless weighted (non-negative) directed graphs $\mathbb{G}^{+}$.
\vfill
For $\Gamma=(V,E,w)\in\mathbb{G}^{+}$, we denote by $V$ the vertex set, by $E$ the edge set, and by $w\colon V\times V\rightarrow\mathbb{R}_{\geq 0}$ the associated weight function.
\vfill
If $(i,j)\in E$, then there is an edge from $i$ to $j$;
the weight of $(i,j)$ is given by $w_{ij}$.
\vfill
We use the convention that $w_{ij}=0$ if and only if $(i,j)\notin E$.
\end{frame}

\begin{frame}{The Laplacian}
The in-degree and out-degree of the vertex $k\in V$ is defined by
\[
d_{i}(k)=\sum_{j}w_{jk}~\text{ and }~d_{o}(k)=\sum_{j}w_{kj},
\]
respectively.
\vfill
Let $D=\diag{d_{o}(1),\ldots,d_{o}(n)}$ denote the out degree matrix and $A=[w_{ij}]_{i,j=1}^{n}$ denote the adjacency matrix.
\vfill
Then, the graph Laplacian is defined by
\[
L=D-A.
\]
\end{frame}

\begin{frame}{Irreducible Matrix}
The matrix $A$ is reducible if there exists a permutation matrix $P$ such that
\[
P^{T}AP = \begin{bmatrix} A_{1} & A_{12} \\ 0 & A_{2}\end{bmatrix}.
\]
\vfill
We say that the matrix $A$ is irreducible if it is not reducible. 
\end{frame}

\begin{frame}{Frobenius Normal Form}
The Frobenius normal form of the Laplacian is
\[
L = \begin{bmatrix} L_{1} & L_{12} & \cdots & L_{1k} \\
				& L_{2} & \cdots & L_{2k} \\
				& & \ddots & \vdots \\
				& & & L_{k} \end{bmatrix}.
\]
\vfill
The blocks $L_{i}$ are irreducible matrices that correspond to the strongly connected components of $\Gamma$, denoted $\Gamma_{i}$.
\vfill
We call $(k-1)$ the degree of reducibility of $L$.
\end{frame}

\begin{frame}{Example}
\begin{minipage}{0.45\textwidth}
The graph $\Gamma$:
\begin{figure}[h]
\centering
\begin{tikzpicture}
	\node[circle, shading=ball, ball color=gray, color=white] (1) at (-2,1) {$1$};
	\node[circle, shading=ball, ball color=gray, color=white] (2) at (-2,-1) {$2$};
	\node[circle, shading=ball, ball color=gray, color=white] (3) at (0,1) {$3$};
	\node[circle, shading=ball, ball color=gray, color=white] (4) at (0,-1) {$4$};
	\node[circle, shading=ball, ball color=gray, color=white] (5) at (2,1) {$5$};
	\node[circle, shading=ball, ball color=gray, color=white] (6) at (2,-1) {$6$};
	
	\draw[gray,->,thick](1) to [in=45,out=315,looseness=1](2);
	\draw[gray,->,thick](2) to [in=225,out=135,looseness=1](1);
	\draw[gray,->,thick](1) to [in=180,out=0,looseness=0](3);
	\draw[gray,->,thick](3) to [in=45,out=315,looseness=1](4);
	\draw[gray,->,thick](4) to [in=225,out=135,looseness=1](3);
	\draw[gray,->,thick](5) to [in=90,out=270,looseness=0](6);\end{tikzpicture}
\end{figure}
\end{minipage}\hfill
\begin{minipage}{0.45\textwidth}
The graph Laplacian:
\centering
\[
L = \begin{bmatrix}
		2 & -1 & -1 & 0 & 0 & 0 \\
		-1 & 1 & 0 & 0 & 0 & 0 \\
		0 & 0 & 1 & -1 & 0 & 0 \\
		0 & 0 & -1 & 1 & 0 & 0 \\
		0 & 0 & 0 & 0 & 1 & -1 \\
		0 & 0 & 0 & 0 & 0 & 0
		\end{bmatrix}
\]
\end{minipage}
\end{frame}

%%%%%%%%%%%%%%%%%%%%%%%%%%%%%%%%%%%%%%%%%%%%%%%%%%%%%%
%								Isolation
%%%%%%%%%%%%%%%%%%%%%%%%%%%%%%%%%%%%%%%%%%%%%%%%%%%%%%
\section{Isolation}

\begin{frame}{Isolated Vertices}
The vertex $i\in V$ is isolated if $w_{ij}=0$ for all $j\in V$.
\vfill
Let $V_{R}\subset V$ denote the subset of vertices that are not isolated
\vfill
Let $L_{R}$ denote the principle submatrix of $L$ whose columns and rows correspond to the vertices in $V_{R}$.
\vfill
Note that $\sigma(L)=\sigma(L_{R})\cup\left\{\abs{V\setminus{V_{R}}}~\text{times the eigenvalue $0$}\right\}$
\end{frame}

\begin{frame}{Isolated Components}
Let $\Gamma=(V,E,w)\in\mathbb{G^{+}}$ and let $\Gamma'=(V',E',w')$ be an induced subgraph of $\Gamma$, i.e., $V'\subseteq V$, $E'=E\cap(V'\times V')$, and $w'=w\restriction_{E'}$.
\vfill
We say that $\Gamma'$ is isolated if $w_{ij}=0$ for all $i\in V'$ and $j\notin V'$.
\vfill
Let $L$ be in its Frobenius normal form, then $\Gamma_{i}$ is isolated if and only if $L_{ij}=0$ for $j=i+1,\ldots,k$.
\end{frame}

\begin{frame}{Lemma 1}
Let $\Gamma\in\mathbb{G}^{+}$ and let $\Gamma'$ be an induced subgraph of $\Gamma$.
If $\Gamma'$ is isolated, then
\[
\sigma(L(\Gamma'))\subseteq\sigma(L(\Gamma)).
\]
\end{frame}

\begin{frame}{Theorem 1 (Taussky)}
A complex $n\times n$ matrix $A$ is non-singular if $A$ is irreducible and
\[
\abs{A_{ii}}\geq\sum\limits_{j\neq i}\abs{A_{ij}},
\]
with equality in at most $(n-1)$ cases.
\end{frame}

\begin{frame}{Lemma 2}
Let $\Gamma\in\mathbb{G}^{+}$ and let $L$ be the graph Laplacian of $\Gamma$ in Frobenius normal form.
\vfill
Then, zero is an eigenvalue (in fact a simple eigenvalue) of $L_{i}$ if and only if $\Gamma_{i}$ is isolated.
\end{frame}

\begin{frame}{Theorem 2}
Let $\Gamma\in\mathbb{G}^{+}$. Then, the following statements are equivalent:
\vfill
\begin{enumerate}[i.]
\item The algebraic multiplicity of the zero eigenvalue of $L$ is equal to $k\in\mathbb{N}$.
\vfill
\item	There exists $k$ isolated strongly connected components of $\Gamma$.
\vfill
\item	The minimum number of directed trees needed to span the whole graph is equal to $k\in\mathbb{N}$.
\end{enumerate}
\end{frame}

%%%%%%%%%%%%%%%%%%%%%%%%%%%%%%%%%%%%%%%%%%%%%%%%%%%%%%
%								Acyclic
%%%%%%%%%%%%%%%%%%%%%%%%%%%%%%%%%%%%%%%%%%%%%%%%%%%%%%
\section{Acyclic}

\begin{frame}{Directed Cycles}
Let $\Gamma\in\mathbb{G}^{+}$. 
A directed cycle in $\Gamma$ is a cycle with all edges oriented in the same direction.
\vfill
A vertex of $\Gamma$ that is contained in at least one directed cycle is called a cyclic vertex.
\vfill
We say that $\Gamma$ is acyclic if none of its vertices are cyclic.
\end{frame}

\begin{frame}{Lemma 3}
The following statements are equivalent:
\vfill
\begin{enumerate}[i.]
\item	The graph $\Gamma\in\mathbb{G}^{+}$ is acyclic.
\vfill
\item	Every strongly connected component of $\Gamma$ consists of exactly one vertex.
\vfill
\item	The graph Laplacian of $\Gamma$ in its Frobenius normal form is upper triangular. 
\end{enumerate}
\end{frame}

\begin{frame}{Theorem 3}
Let $\Gamma\in\mathbb{G}^{+}$ and let $L$ be the graph Laplacian of $\Gamma$.
Then,
\[
\sigma(L)=\left\{d_{o}(1),\ldots,d_{o}(n)\right\}
\]
if and only if $\Gamma$ is acyclic. 
\end{frame}

\begin{frame}{Corollary 1}
Let $\Gamma\in\mathbb{G}^{+}$ have binary weights and let $L$ be the graph Laplacian of $\Gamma$.
\vfill
Then, $\Gamma$ is a perfect dominance graph if and only if
\[
\sigma(L)=\left\{d_{o}(1),\ldots,d_{o}(n)\right\}
\]
and there exists a re-ordering of the vertices such that $d_{o}(i)=n-i$ for $i=1,\ldots,n$.
\end{frame}

\begin{frame}{Corollary 2}
Let $\Gamma\in\mathbb{G}^{+}$ have binary weights and let $L$ be the graph Laplacian of $\Gamma$.
\vfill
Then, $\Gamma$ is a perfect dominance graph if and only if there exists a permutation matrix $P$ such that the eigenpairs $(\lambda_{i},v_{i})$ of $L$ satisfy
\[
\lambda_{i}=n-i
\]
and
\[
Pv_{i}=\sum_{k=1}^{i}e_{i},
\]
for $i=1,\ldots,n$.
\end{frame}

\begin{frame}{Rankability Measure}
Let $\Gamma\in\mathbb{G}^{+}$ have binary weights and let $L$ be the graph Laplacian of $\Gamma$.
\vfill
Denote the eigenvalues of $L$ and the out-degrees of $\Gamma$ by
\begin{align*}
\sigma &= \left\{\lambda_{1},\ldots,\lambda_{n}\right\}, \\
\delta &= \left\{d_{o}(1),\ldots,d_{o}(n)\right\},
\end{align*}
respectively.
\vfill
The rankability measure of $\Gamma$ is defined by
\[
\frac{d(\sigma,s)+d(\delta,s)}{2(n-1)},
\]
where $s=\left\{n-1,\ldots,1,0\right\}$ and $d$ is the Hausdorff distance between two sets.
\end{frame}

\begin{frame}{Example}
\centering
\begin{tabular}{c | c}
Graph & Rankability Measure \\
\hline
Dominance Graph & 0.000 \\
Perturbed Dominance Graph & 0.062 \\
Perturbed Random Graph & 0.223 \\
Nearly Disconnected & 0.400 \\
Random & 0.565 \\
Cyclic & 0.700 \\
Completely Connected & 0.800 \\
Empty Graph & 1.000
\end{tabular}
\end{frame}

\begin{frame}{Example}
\centering
\begin{tabular}{cc}
\begin{tikzpicture}[scale=0.6]
	\node[circle, shading=ball, ball color=gray, color=white] (1) at (-1,2) {$1$};
	\node[circle, shading=ball, ball color=gray, color=white] (2) at (-1,0) {$2$};
	\node[circle, shading=ball, ball color=gray, color=white] (3) at (-1,-2) {$3$};
	\node[circle, shading=ball, ball color=gray, color=white] (4) at (1,2) {$4$};
	\node[circle, shading=ball, ball color=gray, color=white] (5) at (1,0) {$5$};
	\node[circle, shading=ball, ball color=gray, color=white] (6) at (1,-2) {$6$};
	
	\draw[gray,->,thick](1) to [out=270,in=90,looseness=0](2);
	\draw[gray,->,thick](2) to [out=270,in=90,looseness=0](3);
	\draw[gray,->,thick](1) to [out=225,in=135,looseness=1](3);
	\draw[gray,->,thick](1) to [out=0,in=180,looseness=0](4);
	\draw[gray,->,thick](4) to [out=270,in=90,looseness=0](5);
	\draw[gray,->,thick](5) to [out=270,in=90,looseness=0](6);
	\draw[gray,->,thick](4) to [out=315,in=45,looseness=1](6);
\end{tikzpicture}
&
\begin{tikzpicture}[scale=0.6]
	\node[circle, shading=ball, ball color=gray, color=white] (1) at (-1,2) {$1$};
	\node[circle, shading=ball, ball color=gray, color=white] (2) at (-1,0) {$2$};
	\node[circle, shading=ball, ball color=gray, color=white] (3) at (-1,-2) {$3$};
	\node[circle, shading=ball, ball color=gray, color=white] (4) at (1,2) {$4$};
	\node[circle, shading=ball, ball color=gray, color=white] (5) at (1,0) {$5$};
	\node[circle, shading=ball, ball color=gray, color=white] (6) at (1,-2) {$6$};
	
	\draw[gray,->,thick](1) to [out=270,in=90,looseness=0](2);
	\draw[gray,->,thick](2) to [out=270,in=90,looseness=0](3);
	\draw[gray,->,thick](1) to [out=225,in=135,looseness=1](3);
	\draw[gray,->,thick](3) to [out=45,in=225,looseness=0](4);
	\draw[gray,->,thick](4) to [out=270,in=90,looseness=0](5);
	\draw[gray,->,thick](5) to [out=270,in=90,looseness=0](6);
	\draw[gray,->,thick](4) to [out=315,in=45,looseness=1](6);
\end{tikzpicture} \\
$\text{rankability} = 0.4$ & $\text{rankability} = 0.6$ \\
\begin{tikzpicture}[scale=0.6]
	\node[circle, shading=ball, ball color=gray, color=white] (1) at (-1,2) {$1$};
	\node[circle, shading=ball, ball color=gray, color=white] (2) at (-1,0) {$2$};
	\node[circle, shading=ball, ball color=gray, color=white] (3) at (-1,-2) {$3$};
	\node[circle, shading=ball, ball color=gray, color=white] (4) at (1,2) {$4$};
	\node[circle, shading=ball, ball color=gray, color=white] (5) at (1,0) {$5$};
	\node[circle, shading=ball, ball color=gray, color=white] (6) at (1,-2) {$6$};
	
	\draw[gray,->,thick](1) to [out=270,in=90,looseness=0](2);
	\draw[gray,->,thick](2) to [out=270,in=90,looseness=0](3);
	\draw[gray,->,thick](1) to [out=225,in=135,looseness=1](3);
	\draw[gray,->,thick](2) to [out=45,in=225,looseness=0](4);
	\draw[gray,->,thick](4) to [out=270,in=90,looseness=0](5);
	\draw[gray,->,thick](5) to [out=270,in=90,looseness=0](6);
	\draw[gray,->,thick](4) to [out=315,in=45,looseness=1](6);
\end{tikzpicture}
&
\begin{tikzpicture}[scale=0.6]
	\node[circle, shading=ball, ball color=gray, color=white] (1) at (-1,2) {$1$};
	\node[circle, shading=ball, ball color=gray, color=white] (2) at (-1,0) {$2$};
	\node[circle, shading=ball, ball color=gray, color=white] (3) at (-1,-2) {$3$};
	\node[circle, shading=ball, ball color=gray, color=white] (4) at (1,2) {$4$};
	\node[circle, shading=ball, ball color=gray, color=white] (5) at (1,0) {$5$};
	\node[circle, shading=ball, ball color=gray, color=white] (6) at (1,-2) {$6$};
	
	\draw[gray,->,thick](1) to [out=270,in=90,looseness=0](2);
	\draw[gray,->,thick](2) to [out=270,in=90,looseness=0](3);
	\draw[gray,->,thick](1) to [out=225,in=135,looseness=1](3);
	\draw[gray,->,thick](4) to [out=270,in=90,looseness=0](5);
	\draw[gray,->,thick](5) to [out=270,in=90,looseness=0](6);
	\draw[gray,->,thick](4) to [out=315,in=45,looseness=1](6);
\end{tikzpicture} \\
$\text{rankability} = 0.6$ & $\text{rankability} = 0.6$
\end{tabular}
\end{frame}

%%%%%%%%%%%%%%%%%%%%%%%%%%%%%%%%%%%%%%%%%%%%%%%%%%%%%%
%								Algebraic Connectivity
%%%%%%%%%%%%%%%%%%%%%%%%%%%%%%%%%%%%%%%%%%%%%%%%%%%%%%
\section{Algebraic Connectivity}

\begin{frame}{Algebraic Connectivity}
We define the algebraic connectivity of $\Gamma\in\mathbb{G}^{+}$ by
\vfill
\[
\alpha(\Gamma) = \min_{x\in S}x^{T}Lx,
\]
\vfill
where $L$ is the graph Laplacian of $\Gamma$ and
\vfill
\[
S = \{x\in\mathbb{R}^{n}\colon~\norm{x}=1,~x\perp e\}.
\]
\vfill
Another useful quantity is
\[
\beta(\Gamma) = \max_{x\in S}x^{T}Lx.
\]
\end{frame}

\begin{frame}{Properties}
\begin{itemize}
\item	$\alpha(\Gamma)$ and $\beta(\Gamma)$ are independent of the ordering of vertices since $S$ is an invariant subspace of permutation matrices.
\vfill
\item	Let $Q$ be an orthonormal matrix whose columns span $S$, then
	\[
	\alpha(\Gamma) = \lambda_{\text{min}}\left(\frac{1}{2}Q^{T}\left(L+L^{T}\right)Q\right)
	\]
and
	\[
	\beta(\Gamma) = \lambda_{\text{max}}\left(\frac{1}{2}Q^{T}\left(L+L^{T}\right)Q\right)
	\]
\end{itemize}
\end{frame}

\begin{frame}{Properties}
\begin{itemize}
\item	If $\Gamma_{1},\Gamma_{2}\in\mathbb{G}^{+}$ have the same vertex set, then
	\[
	\alpha(\Gamma_{1})+\alpha(\Gamma_{2}) \leq \alpha(\Gamma_{1}\cup\Gamma_{2}) \leq \beta(\Gamma_{1}\cup\Gamma_{2}) \leq \beta(\Gamma_{1})+\beta(\Gamma_{2}).
	\]
\vfill
\item	Let $\Gamma_{1}\times \Gamma_{2}$ be the cartesian product of two graphs $\Gamma_{1}$ and $\Gamma_{2}$, then
	\[
	\alpha(\Gamma_{1}\times\Gamma_{2})\leq\min(\alpha(\Gamma_{1}),\alpha(\Gamma_{2}))\leq\max(\beta(\Gamma_{1}),\beta(\Gamma_{2}))\leq\beta(\Gamma_{1}\times\Gamma_{2}).
	\]
\end{itemize}
\end{frame}

\begin{frame}{Properties}
\begin{itemize}
\item Let $\Gamma\in\mathbb{G}^{+}$ with graph Laplacian $L$. Then,
	\[
	\lambda_{1}\left(\frac{1}{2}(L+L^{T})\right)\leq\alpha(\Gamma)\leq\lambda_{2}\left(\frac{1}{2}(L+L^{T})\right)
	\]
	and
	\[
	\lambda_{n-1}\left(\frac{1}{2}(L+L^{T})\right)\leq\beta(\Gamma)\leq\lambda_{n}\left(\frac{1}{2}(L+L^{T})\right)
	\]
\item	Let $i,j$ be non-adjacent vertices of $\Gamma\in\mathbb{G}^{+}$. Then,
	\[
	\alpha(\Gamma)\leq\frac{1}{2}\left(d_{o}(i)+d_{o}(j)\right)\leq\beta(\Gamma).
	\]
\end{itemize}
\end{frame}

\begin{frame}{Example}
\begin{minipage}{0.45\textwidth}
\begin{tikzpicture}
	\node[circle, shading=ball, ball color=gray, color=white] (1) at (-1,2) {$1$};
	\node[circle, shading=ball, ball color=gray, color=white] (2) at (-1,0) {$2$};
	\node[circle, shading=ball, ball color=gray, color=white] (3) at (-1,-2) {$3$};
	\node[circle, shading=ball, ball color=gray, color=white] (4) at (1,2) {$4$};
	\node[circle, shading=ball, ball color=gray, color=white] (5) at (1,0) {$5$};
	\node[circle, shading=ball, ball color=gray, color=white] (6) at (1,-2) {$6$};
	
	\draw[gray,->,thick](1) to [out=270,in=90,looseness=0](2);
	\draw[gray,->,thick](2) to [out=270,in=90,looseness=0](3);
	\draw[gray,->,thick](1) to [out=225,in=135,looseness=1](3);
	\draw[gray,->,thick](4) to [out=270,in=90,looseness=0](5);
	\draw[gray,->,thick](5) to [out=270,in=90,looseness=0](6);
	\draw[gray,->,thick](4) to [out=315,in=45,looseness=1](6);
\end{tikzpicture}
\end{minipage}\hfill
\begin{minipage}{0.45\textwidth}
Eigenvalues of the Laplacian:
\[
\sigma(L)=\{2,1,0,2,1,0\}
\]
\vfill
Spectral Rankability: 0.6 \\
\vfill
Algebraic Connectivity: -0.389 \\
\vfill
AC Rankability: 1.0 \\
\end{minipage}
\end{frame}

\begin{frame}{Example}
\begin{minipage}{0.45\textwidth}
\begin{tikzpicture}
	\node[circle, shading=ball, ball color=gray, color=white] (1) at (-1,2) {$1$};
	\node[circle, shading=ball, ball color=gray, color=white] (2) at (-1,0) {$2$};
	\node[circle, shading=ball, ball color=gray, color=white] (3) at (-1,-2) {$3$};
	\node[circle, shading=ball, ball color=gray, color=white] (4) at (1,2) {$4$};
	\node[circle, shading=ball, ball color=gray, color=white] (5) at (1,0) {$5$};
	\node[circle, shading=ball, ball color=gray, color=white] (6) at (1,-2) {$6$};
	
	\draw[gray,->,thick](1) to [out=270,in=90,looseness=0](2);
	\draw[gray,->,thick](2) to [out=270,in=90,looseness=0](3);
	\draw[gray,->,thick](1) to [out=225,in=135,looseness=1](3);
	\draw[gray,->,thick](2) to [out=45,in=225,looseness=0](4);
	\draw[gray,->,thick](4) to [out=270,in=90,looseness=0](5);
	\draw[gray,->,thick](5) to [out=270,in=90,looseness=0](6);
	\draw[gray,->,thick](4) to [out=315,in=45,looseness=1](6);
\end{tikzpicture}
\end{minipage}\hfill
\begin{minipage}{0.45\textwidth}
Eigenvalues of the Laplacian:
\[
\sigma(L)=\{2,2,0,2,1,0\}
\]
\vfill
Spectral Rankability: 0.6 \\
\vfill
Algebraic Connectivity: -0.303 \\
\vfill
AC Rankability: 0.929 \\
\end{minipage}
\end{frame}

\begin{frame}{Example}
\begin{minipage}{0.45\textwidth}
\begin{tikzpicture}
	\node[circle, shading=ball, ball color=gray, color=white] (1) at (-1,2) {$1$};
	\node[circle, shading=ball, ball color=gray, color=white] (2) at (-1,0) {$2$};
	\node[circle, shading=ball, ball color=gray, color=white] (3) at (-1,-2) {$3$};
	\node[circle, shading=ball, ball color=gray, color=white] (4) at (1,2) {$4$};
	\node[circle, shading=ball, ball color=gray, color=white] (5) at (1,0) {$5$};
	\node[circle, shading=ball, ball color=gray, color=white] (6) at (1,-2) {$6$};
	
	\draw[gray,->,thick](1) to [out=270,in=90,looseness=0](2);
	\draw[gray,->,thick](2) to [out=270,in=90,looseness=0](3);
	\draw[gray,->,thick](1) to [out=225,in=135,looseness=1](3);
	\draw[gray,->,thick](3) to [out=45,in=225,looseness=0](4);
	\draw[gray,->,thick](4) to [out=270,in=90,looseness=0](5);
	\draw[gray,->,thick](5) to [out=270,in=90,looseness=0](6);
	\draw[gray,->,thick](4) to [out=315,in=45,looseness=1](6);
\end{tikzpicture}
\end{minipage}\hfill
\begin{minipage}{0.45\textwidth}
Eigenvalues of the Laplacian:
\[
\sigma(L)=\{2,1,1,2,1,0\}
\]
\vfill
Spectral Rankability: 0.6 \\
\vfill
Algebraic Connectivity: -0.068 \\
\vfill
AC Rankability: 0.738 \\
\end{minipage}
\end{frame}

\end{document}