% document class and packages
\documentclass{beamer}
\usepackage{adjustbox}
\usepackage{algorithm,algorithmic}
\usepackage{amsmath}
\usepackage{amssymb}
\usepackage{color, colortbl}
\usepackage{graphicx}
\usepackage{hyperref}
\usepackage{pgfplots}
\pgfplotsset{compat=1.14}
\usepackage{tikz}

% indent for algorithm pseudo-code
\newlength\myindent
\setlength\myindent{1em}
\newcommand\bindent{%
  \begingroup
  \setlength{\itemindent}{\myindent}
  \addtolength{\algorithmicindent}{\myindent}
}
\newcommand\eindent{\endgroup}

% new commands and operators
\newcommand\diag[1]{\operatorname{diag}\left(#1\right)}
\newcommand\abs[1]{\left|#1\right|}
\newcommand\norm[1]{\left\Vert#1\right\Vert}
\newcommand{\iu}{{i\mkern1mu}}
\DeclareMathOperator{\func}{function}
\DeclareMathOperator{\specR}{specR}
\DeclareMathOperator{\connR}{connR}

% 
\newtheorem{proposition}[theorem]{Proposition}

% remove figure caption prefix
\setbeamertemplate{caption}{\raggedright\insertcaption\par}

% hyperlinks setup
\hypersetup{colorlinks,breaklinks,
	urlcolor=[rgb]{0,0.75,1},
	linkcolor=[rgb]{0.75,0.75,0.75}}

% empty navigation symbols
\beamertemplatenavigationsymbolsempty

% remove navigation dots on miniframes
\makeatletter
\def\beamer@writeslidentry{\clearpage\beamer@notesactions}
\makeatother

% Use Theme
\usetheme{Warsaw}
\useoutertheme[footline=authortitle]{miniframes}
\useinnertheme[shadow=true]{rounded}

% Colors
\definecolor{black}{RGB}{0, 0, 0} % (primary, black)
\definecolor{lblue}{RGB}{102, 178, 255} % (secondary, light blue)
\definecolor{lgreen}{RGB}{102, 255, 178} %(tertiary, light green)
\definecolor{lsilver}{RGB}{224,224,224} % (text, light silver)
\definecolor{gray}{RGB}{128,128,128} % (graph node shade, gray)
\definecolor{white}{RGB}{255,255,255} % (graph node text, white)

% Beamer Colors
\setbeamercolor{palette primary}{bg=black,fg=lsilver}
\setbeamercolor{palette secondary}{bg=lblue,fg=lsilver}
\setbeamercolor{palette tertiary}{bg=black,fg=lsilver}
\setbeamercolor{structure}{fg=black} % itemize, enumerate, etc
\setbeamercolor{frametitle}{fg=black}

% Transparency for itemized listing
%\setbeamercovered{transparent}

% Title Page
\title{Update: Algebraic Connectivity}
\author{Thomas R. Cameron}
\institute{Davidson College}
\date{May 28, 2019}

\begin{document}
% Title Frame
\begin{frame}
	\titlepage
\end{frame}

% Outline
%\AtBeginSection[]{
 %\frame<beamer>{
  %\frametitle{Outline}   
  %\tableofcontents[currentsection]
 %}
%}

%%%%%%%%%%%%%%%%%%%%%%%%%%%%%%%%%%%%%%%%%%%%%%%%%%%%%%
%								Measures of Rankability
%%%%%%%%%%%%%%%%%%%%%%%%%%%%%%%%%%%%%%%%%%%%%%%%%%%%%%
\section{Measures of Rankability}

\begin{frame}{Algebraic Connectivity}
Let $\Gamma$ be a directed graph and $L$ be the graph Laplacian.
\vfill
Define
\[
S=\{x\in\mathbb{R}^{n}\colon x\perp e, \norm{x}=1\}.
\]
\vfill
Then, the algebraic connectivity of $\Gamma$ is
\[
\alpha(\Gamma) = \min_{x\in S}x^{T}Lx,
\]
\vfill
and a related quantity is
\[
\beta(\Gamma) = \max_{x\in S}x^{T}Lx.
\]
\end{frame}

\begin{frame}{Results}
Let $\Gamma$ be a perfect dominance graph on $n$ vertices.
Then,
\vfill
\begin{itemize}
\item	$\alpha(\Gamma)+\beta(\Gamma)=n$.
\vfill
\item $0<\alpha(\Gamma)\leq1$ and $n-1\leq \beta(\Gamma) <n$.
\end{itemize}
\end{frame}

%%%%%%%%%%%%%%%%%%%%%%%%%%%%%%%%%%%%%%%%%%%%%%%%%%%%%%
%								Big East Data
%%%%%%%%%%%%%%%%%%%%%%%%%%%%%%%%%%%%%%%%%%%%%%%%%%%%%%
\section{Big East Data}

\begin{frame}{Snapshot of Results}
\centering
\begin{tabular}{|| c | c | c | c || c | c | c | c ||}
\hline
Year & $\specR$ & $\alpha R$ & $\beta R$ & Year & $\specR$ & $\alpha R$ & $\beta R$ \\
\hline\hline
1995 & 0.143 & 0.006 & 0.408 & 2004 & 0.339 & 0.669 & 0.993 \\ 
1996 & 0.143 & 0.001 & 0.408 & 2005 & 0.162 & 0.095 & 0.065 \\
1997 & 0.185 & 0.153 & 0.565 & 2006 & 0.195 & 0.680 & 0.543 \\
1998 & 0.183 & 0.093 & 0.530 & 2007 & 0.316 & 1.000 & 1.000 \\
1999 & 0.143 & 0.902 & 0.012 & 2008 & 0.195 & 0.680 & 0.531 \\
2000 & 0.143 & 0.013 & 0.008 & 2009 & 0.143 & 0.680 & 0.058 \\
2001 & 0.143 & 0.003 & 0.006 & 2010 & 0.292 & 0.947 & 1.000 \\
2002 & 0.143 & 0.080 & 0.003 & 2011 & 0.286 & 0.680 & 1.000 \\
2003 & 0.143 & 0.090 & 0.408 & 2012 & 0.286 & 0.912 & 1.000 \\
\hline
\end{tabular}
\vfill
\fcolorbox{black}{lblue}{\rule{0pt}{6pt}\rule{6pt}{0pt}}\quad Most Rankable\\
\vfill
\fcolorbox{black}{lsilver}{\rule{0pt}{6pt}\rule{6pt}{0pt}}\quad Least Rankable\\
\end{frame}

\end{document}