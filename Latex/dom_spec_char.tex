\documentclass{article}
\usepackage{amsmath}
\usepackage{amsthm}
\usepackage{amssymb}
\usepackage{enumerate}
\usepackage[margin=1.00in]{geometry}

\newtheorem{theorem}{Theorem}[section]
\newtheorem{corollary}[theorem]{Corollary}
\newtheorem{example}[theorem]{Example}

\DeclareMathOperator{\md}{md}



\title{A Note on Rankability:\\
\emph{\large{Spectral Characterization of the Dominance Graph}}}
\author{Thomas R. Cameron}
\date{March 2019}

\begin{document}
\maketitle
\abstract{
This note is an organization of the author's development of a spectral characterization of dominance graphs and its application to the problem of rankability. 
}

%%%%%%%%%%%%%%%%%%%%%%%%%%%%%%%%%%%%%%%%%%%%%%%%%%%%%%
%                                    				Introduction
%%%%%%%%%%%%%%%%%%%%%%%%%%%%%%%%%%%%%%%%%%%%%%%%%%%%%%
\section{Introduction}	

%%%%%%%%%%%%%%%%%%%%%%%%%%%%%%%%%%%%%%%%%%%%%%%%%%%%%%
%                                    				Spectral Characterization
%%%%%%%%%%%%%%%%%%%%%%%%%%%%%%%%%%%%%%%%%%%%%%%%%%%%%%
\section{Spectral Characterization}
Let $G=(V,E)$ be a directed graph, where $V\subseteq\mathbb{N}$ and $(i,j)\in E$ if vertex $i$ points to vertex $j$.
We define the adjacency matrix of $G$ to be $A(G)=[a_{ij}]$, where $a_{ij}=1$ if $(i,j)\in E$ and $a_{ij}=0$ otherwise. 
Similarly, the degree matrix of $G$, denoted $D(G)=[d_{ij}]$, is defined by $d_{ij}=\deg(i)$ (out degree) if $i=j$ and $d_{ij}=0$ otherwise.

We define the graph Laplacian by 
\begin{equation}\label{eq:laplacian}
L(G)=D(G)-A(G).
\end{equation}
Note that for undirected graphs this definition is standard. 
However, the spectral approach for directed graphs has not been well developed, and the definition of the graph Laplacian is not standard.
For instance, another definition for the Laplacian of a directed graph is given in~\cite{Chung2005}.

For undirected graphs, there is a deep connection between the properties of a graph and the spectral properties of the Laplacian. 
We will see a similar phenomena for directed graphs and the graph Laplacian as defined in~\eqref{eq:laplacian}.
We motivate our main result with the following example.

\begin{example}
Let $G$ denote the dominance graph on $6$ vertices shown in Figure 2 of~\cite{Anderson2018}. 
Note that the Laplacian of this graph is 
\[
L(G)=
\begin{bmatrix}
5 & -1 & -1 & -1 & -1 & -1 \\
0 & 4 & -1 & -1 & -1 & -1 \\
0 & 0 & 3 & -1 & -1 & -1 \\
0 & 0 & 0 & 2 & -1 & -1 \\
0 & 0 & 0 & 0 & 1 & -1 \\
0 & 0 & 0 & 0 & 0 & 0
\end{bmatrix}.
\]
It is clear that the eigenvalues of $L(G)$ are $\lambda_{k}=6-k$, for $k=1,\ldots,6$.
Moreover, the corresponding eigenvectors are $v_{1}=e_{1}$, and $v_{k}=v_{k-1}+e_{k}$, for $k=2,\ldots,6$.

Suppose that we reorder the vertices of the graph $G$ to obtain the isomorphic graph $\hat{G}$ that corresponds to the ranking $[1,3,4,2,5,6]$.
The resulting graph Laplacian is as follows:
\[
L(\hat{G})=
\begin{bmatrix}
 5 & -1 & -1 & -1 & -1 & -1 \\
 0 & 3 & -1 & 0 & -1 & -1 \\
 0 & 0 & 2 & 0 & -1 & -1 \\
 0 & -1 & -1 & 4 & -1 & -1 \\
 0 & 0 & 0 & 0 & 1 & -1 \\
 0 & 0 & 0 & 0 & 0 & 0 \\
\end{bmatrix}.
\]
Note that $L(\hat{G})=P^{T}L(G)P$, where $P$ is the permutation matrix (column representation) corresponding to the inverse permutation $\pi^{-1}$, where
\[
\pi=
\left(
\begin{array}{cccccc}
1 & 2 & 3 & 4 & 5 & 6 \\
1 & 3 & 4 & 2 & 5 & 6
\end{array}
\right).
\]
Clearly the eigenvalues of $L(G)$ and $L(\hat{G})$ are the same, and the eigenvectors of $L(G)$ and $L(\hat{G})$ are closed related.
In particular, if the eigenvectors of $L(\hat{G})$ are $\{\hat{v}_{1},\ldots,\hat{v}_{6}\}$, then $P\hat{v}_{k}=v_{k}$ for $k=1,\ldots,m$. 
\end{example}

In Example 2.1, there was nothing special about the number of vertices or the reordering of the vertices. 
This implies a result that is not too surprising: 
Let $G$ be any dominance graph on the vertices $\{1,2,\ldots,n\}$. 
Then, the eigenvalues of $L(G)$ are $\lambda_{k}=n-k$ for $k=1,\ldots,n$, and there exists a permutation matrix $P$ such that
the corresponding eigenvectors $v_{1},\ldots,v_{n}$ of $L(G)$ satisfy $Pv_{1}=e_{1}$ and $Pv_{k}=Pv_{k-1}+e_{k}$ for $k=2,\ldots,n$.
What is surprising is that not only is this spectral information necessary for dominance graphs it is also sufficient. 

\begin{theorem}\label{thm:spec-char}
Let $G$ be a graph on the vertices $\{1,2,\ldots,n\}$.
Then, $G$ is a dominance graph if and only if the eigenvalues of $L(G)$ are $\lambda_{k}=n-k$ for $k=1,\ldots,n$ and the corresponding eigenvectors $v_{1},\ldots,v_{n}$ of $L(G)$ satisfy $Pv_{1}=e_{1}$ and $Pv_{k}=Pv_{k-1}+e_{k}$ for $k=2,\ldots,n$ and some permutation matrix $P$. 
\end{theorem}
\begin{proof}
Suppose that $G$ is a dominance graph corresponding to the ranking $[r_{1},\ldots,r_{n}]$.
Let $\pi$ be a permutation defined by $\pi(1)=r_{1},\ldots,\pi(n)=r_{n}$, and also let $P$ denote the permutation matrix corresponding to $\pi$. 
Then,
\[
PL(G)P^{T}=
\begin{bmatrix}
(n-1) & -1 & -1 & \cdots & -1 \\
0 & (n-2) & -1 & \cdots & -1 \\
\vdots & & \ddots & & \vdots \\
0 & 0 & \cdots & 1 & -1 \\
0 & 0 & \cdots & 0 & 0
\end{bmatrix}.
\]
It is clear that the eigenvalues of $L(G)$ are $\lambda_{k}=(n-k)$ for $k=1,\ldots,n$.
Furthermore, note that $PL(G)P^{T}e_{1}=(n-1)e_{1}$, $PL(G)P^{T}(e_{1}+e_{2})=(n-2)(e_{1}+e_{2})$, and so on.
Therefore, the corresponding eigenvectors $v_{1},\ldots,v_{n}$ of $L(G)$ satisfy $Pv_{1}=e_{1}$ and $Pv_{k}=Pv_{k-1}+e_{k}$ for $k=2,\ldots,n$.

Conversely, suppose that $G$ is a graph such that its graph Laplacian $L(G)$ has eigenvalues $\lambda_{k}=n-k$ for $k=1,\ldots,n$ and corresponding eigenvectors $v_{1},\ldots,v_{n}$ that satisfy $Pv_{1}=e_{1}$ and $Pv_{k}=Pv_{k-1}+e_{k}$ for $k=2,\ldots,n$ and some permutation matrix $P$.
Then, for instance, $L(G)v_{1}=(n-1)v_{1}$, i.e., $PL(G)P^{T}e_{1}=(n-1)e_{1}$.

Let $\pi$ be the permutation corresponding to the matrix $P$.
If we write out $L(G)$ in terms of the adjacency matrix and degree matrix, then we have
\[
PD(G)P^{T}e_{1}-PA(G)P^{T}e_{1}=(n-1)e_{1}.
\]
Since the diagonal entries of $A(G)$ and, therefore, $PA(G)P^{T}$ cannot be zero, it follows that $\deg(\pi(1))=(n-1)$ and the first column of the adjacency matrix $PA(G)P^{T}$ is the zero vector.
Similarly, the second eigenvector equation $L(G)v_{2}=(n-2)v_{2}$ can be written as
\[
PD(G)P^{T}(e_{1}+e_{2})-PA(G)P^{T}(e_{1}+e_{2})=(n-2)(e_{1}+e_{2}).
\]
Again, since the diagonal entries of $PA(G)P^{T}$ cannot be zero, it follows that $\deg(\pi(2))=(n-2)$ and the second column of $PA(G)P^{T}$ is equal to $e_{1}$.

We continue in this fashion, thus showing that $\deg(\pi(k))=(n-k)$ for $k=1,\ldots,n$ and the columns of $PA(G)P^{T}$ are the zero vector, $e_{1}$, $(e_{1}+e_{2})$, and so on until the last column which is equal to $(e_{1}+e_{2}+\cdots+e_{n-1})$.
Therefore, $G$ is isomorphic to the dominance graph with ranking $[1,2,\ldots,n]$. 
\end{proof}

%%%%%%%%%%%%%%%%%%%%%%%%%%%%%%%%%%%%%%%%%%%%%%%%%%%%%%
%                                    				Rankability Metric
%%%%%%%%%%%%%%%%%%%%%%%%%%%%%%%%%%%%%%%%%%%%%%%%%%%%%%
\section{Rankability Metric}
The rankability of data refers to a dataset's inherent ability to produce a meaningful ranking of its items. 
In~\cite{Anderson2018}, a rankability measure is proposed based on the distance to the nearest dominance graph.
The distance is measured as the number of changes (edges deleted or added) that would need to be made in order to obtain a dominance graph.
In this section, we propose that the distance be measured based on the spectral characterization of dominance graphs in Theorem~\ref{thm:spec-char}.

Let $G$ be a graph on $n$ vertices.
Denote by $\lambda_{1},\ldots,\lambda_{n}$ the eigenvalues of $L(G)$ and by $v_{1},\ldots,v_{n}$ the corresponding eigenvectors. 
We define the matching distance between the eigenvalues of $L(G)$ and the eigenvalues of a dominance graph by
\[
\md_{\lambda}(G) := \min_{\pi}\left\{\max_{i}\left\{ |\lambda_{pi(i)}-(n-i)| \right\}\right\},
\]
where $\pi$ is taken over all possible permutations of the spectrum $\sigma(L(G))$ and $i$ is taken over the integers $\{1,\ldots,n\}$.




%%%%%%%%%%%%%%%%%%%%%%%%%%%%%%%%%%%%%%%%%%%%%%%%%%%%%%%%
%						References
%%%%%%%%%%%%%%%%%%%%%%%%%%%%%%%%%%%%%%%%%%%%%%%%%%%%%%%%
\bibliographystyle{siam}
\bibliography{Bibliography}


\end{document}