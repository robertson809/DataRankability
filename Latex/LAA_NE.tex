% document class and packages
\documentclass{beamer}
%\usepackage{adjustbox}
%\usepackage{algorithm,algorithmic}
%\usepackage{amsmath}
%\usepackage{amssymb}
%\usepackage{graphicx}
%\usepackage{hyperref}
\usepackage{color, colortbl}
\usepackage{pgfplots}
\pgfplotsset{compat=newest}
\usetikzlibrary{arrows,decorations.markings}

\tikzset{myptr/.style={decoration={markings,mark=at position 1 with %
    {\arrow[scale=1.5,>=stealth]{>}}},postaction={decorate}}}

% indent for algorithm pseudo-code
\newlength\myindent
\setlength\myindent{1em}
\newcommand\bindent{%
  \begingroup
  \setlength{\itemindent}{\myindent}
  \addtolength{\algorithmicindent}{\myindent}
}
\newcommand\eindent{\endgroup}

% remove figure caption prefix
\setbeamertemplate{caption}{\raggedright\insertcaption\par}

% hyperlinks setup
\hypersetup{colorlinks,breaklinks,
	urlcolor=[rgb]{0,0.75,1},
	linkcolor=[rgb]{0.75,0.75,0.75}}

% empty navigation symbols
\beamertemplatenavigationsymbolsempty

% remove navigation dots on miniframes
\makeatletter
\def\beamer@writeslidentry{\clearpage\beamer@notesactions}
\makeatother

% Use Theme
\usetheme{Warsaw}
\useoutertheme[footline=authortitle]{miniframes}
\useinnertheme[shadow=true]{rounded}

% Colors
\definecolor{black}{RGB}{0,0,0} % black
\definecolor{dgreen}{RGB}{0,73,73} % dark green
\definecolor{lgreen}{RGB}{0,146,146} % light green
\definecolor{dpink}{RGB}{255,109,182} % dark pink
\definecolor{lpink}{RGB}{255,182,119} % light pink
\definecolor{dpurple}{RGB}{73,0,146} % dark purple
\definecolor{dblue}{RGB}{0,109,219} % dark blue
\definecolor{lpurple}{RGB}{182,109,255} % light purple
\definecolor{lblue}{RGB}{109,182,255} % light blue
\definecolor{pblue}{RGB}{182,219,255} % powder blue
\definecolor{dred}{RGB}{146,0,0} % dark red
\definecolor{brown}{RGB}{146,73,0} % brown
\definecolor{orange}{RGB}{219,209,0} % orange
\definecolor{bgreen}{RGB}{36,255,36} % bright green
\definecolor{yellow}{RGB}{255,255,109} % yellow

% New Commands
\newcommand\proj[2]{\textrm{proj}_{\textbf{#2}}{\textbf{#1}}}
\newcommand\abs[1]{\left|#1\right|}
\newcommand\norm[1]{\left\Vert#1\right\Vert}
\newcommand\vv[2]{\left\langle #1,#2\right\rangle}
\newcommand\vvv[3]{\left\langle #1,#2,#3\right\rangle}
\DeclareMathOperator{\specr}{specR}
\DeclareMathOperator{\edger}{edgeR}

% Beamer Colors
\setbeamercolor{palette primary}{bg=black,fg=black!10}
\setbeamercolor{palette secondary}{bg=lblue,fg=black!10}
\setbeamercolor{palette tertiary}{bg=black,fg=black!10}
\setbeamercolor{structure}{fg=black}
\setbeamercolor{frametitle}{fg=black}

% Title Page
\title{LAA Numerical Experiments}
\author{Thomas R. Cameron}
\institute{Davidson College}
\date{\today}

\begin{document}
% Title Frame
\begin{frame}
	\titlepage
\end{frame}

\begin{frame}{SpecR vs EdgeR}
\resizebox{1.0\textwidth}{!}{% SIMOD Graphs
\begin{tabular}{cccc}
\textbf{Complete dominance} & \textbf{Perturbed Dominance} & \textbf{Perturbed Random C} & \textbf{Nearly Disconnected} \\
\resizebox{0.2\textwidth}{!}{% Complete dominance
\begin{tikzpicture}
	\node[circle, shading=ball, ball color=black!75, color=white] (1) at (0,2) {\textbf{1}};
	\node[circle, shading=ball, ball color=black!75, color=white] (2) at (1,1) {\textbf{2}};
	\node[circle, shading=ball, ball color=black!75, color=white] (3) at (1,0) {\textbf{3}};
	\node[circle, shading=ball, ball color=black!75, color=white] (4) at (0,-1) {\textbf{4}};
	\node[circle, shading=ball, ball color=black!75, color=white] (5) at (-1,0) {\textbf{5}};
	\node[circle, shading=ball, ball color=black!75, color=white] (6) at (-1,1) {\textbf{6}};
	
	% vertex 1
	\draw[black!75,->,thick](1) to [out=330,in=135,looseness=0](2);
	\draw[black!75,->,thick](1) to [out=295,in=135,looseness=0](3);
	\draw[black!75,->,thick](1) to [out=270,in=90,looseness=0](4);
	\draw[black!75,->,thick](1) to [out=250,in=45,looseness=0](5);
	\draw[black!75,->,thick](1) to [out=225,in=45,looseness=0](6);
	% vertex 2
	\draw[black!75,->,thick](2) to [out=270,in=90,looseness=0](3);
	\draw[black!75,->,thick](2) to [out=240,in=60,looseness=0](4);
	\draw[black!75,->,thick](2) to [out=225,in=15,looseness=0](5);
	\draw[black!75,->,thick](2) to [out=180,in=0,looseness=0](6);
	% vertex 3
	\draw[black!75,->,thick](3) to [out=225,in=45,looseness=0](4);
	\draw[black!75,->,thick](3) to [out=180,in=0,looseness=0](5);
	\draw[black!75,->,thick](3) to [out=165,in=330,looseness=0](6);
	% vertex 4
	\draw[black!75,->,thick](4) to [out=135,in=315,looseness=0](5);
	\draw[black!75,->,thick](4) to [out=115,in=315,looseness=0](6);
	% vertex 5
	\draw[black!75,->,thick](5) to [out=90,in=270,looseness=0](6);
	% vertex 6
\end{tikzpicture}%
}\hfill
&
\resizebox{0.2\textwidth}{!}{% Perturbed Dominance
\begin{tikzpicture}
	\node[circle, shading=ball, ball color=black!75, color=white] (1) at (0,2) {\textbf{1}};
	\node[circle, shading=ball, ball color=black!75, color=white] (2) at (1,1) {\textbf{2}};
	\node[circle, shading=ball, ball color=black!75, color=white] (3) at (1,0) {\textbf{3}};
	\node[circle, shading=ball, ball color=black!75, color=white] (4) at (0,-1) {\textbf{4}};
	\node[circle, shading=ball, ball color=black!75, color=white] (5) at (-1,0) {\textbf{5}};
	\node[circle, shading=ball, ball color=black!75, color=white] (6) at (-1,1) {\textbf{6}};
	
	% vertex 1
	\draw[black!75,->,thick](1) to [out=330,in=135,looseness=0](2);
	\draw[black!75,->,thick](1) to [out=295,in=135,looseness=0](3);
	\draw[black!75,->,thick](1) to [out=270,in=90,looseness=0](4);
	\draw[black!75,->,thick](1) to [out=250,in=45,looseness=0](5);
	\draw[black!75,->,thick](1) to [out=225,in=45,looseness=0](6);
	% vertex 2
	\draw[black!75,->,thick](2) to [out=240,in=60,looseness=0](4);
	\draw[black!75,->,thick](2) to [out=225,in=15,looseness=0](5);
	\draw[black!75,->,thick](2) to [out=180,in=0,looseness=0](6);
	% vertex 3
	\draw[black!75,->,thick](3) to [out=135,in=295,looseness=0](1);
	\draw[black!75,->,thick](3) to [out=225,in=45,looseness=0](4);
	\draw[black!75,->,thick](3) to [out=180,in=0,looseness=0](5);
	\draw[black!75,->,thick](3) to [out=165,in=330,looseness=0](6);
	% vertex 4
	\draw[black!75,->,thick](4) to [out=135,in=315,looseness=0](5);
	\draw[black!75,->,thick](4) to [out=115,in=315,looseness=0](6);
	% vertex 5
	\draw[black!75,->,thick](5) to [out=90,in=270,looseness=0](6);
	% vertex 6
\end{tikzpicture}%
}\hfill
&
\resizebox{0.2\textwidth}{!}{% Perturbed Random C
\begin{tikzpicture}
	\node[circle, shading=ball, ball color=black!75, color=white] (1) at (0,2) {\textbf{1}};
	\node[circle, shading=ball, ball color=black!75, color=white] (2) at (1,1) {\textbf{2}};
	\node[circle, shading=ball, ball color=black!75, color=white] (3) at (1,0) {\textbf{3}};
	\node[circle, shading=ball, ball color=black!75, color=white] (4) at (0,-1) {\textbf{4}};
	\node[circle, shading=ball, ball color=black!75, color=white] (5) at (-1,0) {\textbf{5}};
	\node[circle, shading=ball, ball color=black!75, color=white] (6) at (-1,1) {\textbf{6}};
	
	% vertex 1
	\draw[black!75,->,thick](1) to [out=330,in=135,looseness=0](2);
	\draw[black!75,->,thick](1) to [out=315,in=120,looseness=0](3);
	\draw[black!75,->,thick](1) to [out=225,in=45,looseness=0](6);
	% vertex 2
	\draw[black!75,->,thick](2) to [out=240,in=60,looseness=0](4);
	\draw[black!75,->,thick](2) to [out=210,in=30,looseness=0](5);
	% vertex 3
	% vertex 4
	\draw[black!75,->,thick](4) to [out=90,in=270,looseness=0](1);
	\draw[black!75,->,thick](4) to [out=60,in=240,looseness=0](2);
	\draw[black!75,->,thick](4) to [out=115,in=315,looseness=0](6);
	% vertex 5
	\draw[black!75,->,thick](5) to [out=45,in=255,looseness=0](1);
	\draw[black!75,->,thick](5) to [out=30,in=210,looseness=0](2);
	% vertex 6
	\draw[black!75,->,thick](6) to [out=45,in=225,looseness=0](1);
	\draw[black!75,->,thick](6) to [out=0,in=180,looseness=0](2);
	\draw[black!75,->,thick](6) to [out=330,in=165,looseness=0](3);
	\draw[black!75,->,thick](6) to [out=270,in=90,looseness=0](5);
\end{tikzpicture}%
}\hfill
&
\resizebox{0.2\textwidth}{!}{% Near Disconnected
\begin{tikzpicture}
	\node[circle, shading=ball, ball color=black!75, color=white] (1) at (0,2) {\textbf{1}};
	\node[circle, shading=ball, ball color=black!75, color=white] (2) at (1,1) {\textbf{2}};
	\node[circle, shading=ball, ball color=black!75, color=white] (3) at (1,0) {\textbf{3}};
	\node[circle, shading=ball, ball color=black!75, color=white] (4) at (0,-1) {\textbf{4}};
	\node[circle, shading=ball, ball color=black!75, color=white] (5) at (-1,0) {\textbf{5}};
	\node[circle, shading=ball, ball color=black!75, color=white] (6) at (-1,1) {\textbf{6}};
	
	% vertex 1
	\draw[black!75,->,thick](1) to [out=330,in=135,looseness=0](2);
	\draw[black!75,->,thick](1) to [out=295,in=135,looseness=0](3);
	\draw[black!75,->,thick](1) to [out=270,in=90,looseness=0](4);
	% vertex 2
	\draw[black!75,->,thick](2) to [out=270,in=90,looseness=0](3);
	% vertex 3
	% vertex 4
	\draw[black!75,->,thick](4) to [out=135,in=315,looseness=0](5);
	\draw[black!75,->,thick](4) to [out=115,in=315,looseness=0](6);
	% vertex 5
	\draw[black!75,->,thick](5) to [out=90,in=270,looseness=0](6);
	% vertex 6
\end{tikzpicture}%
}\\
$\specr = 1.0000$ & $\specr = 0.9382$ & $\specr = 0.8202$ & $\specr = 0.6000$ \\
$\edger = 1.0000$ & $\edger = 0.9994$ & $\edger = 0.9987$ & $\edger = 0.9926$ \\
& & & \\
\textbf{Random} & \textbf{Cycle} & \textbf{Completely Connected} & \textbf{Empty} \\
\resizebox{0.2\textwidth}{!}{% Random
\begin{tikzpicture}
	\node[circle, shading=ball, ball color=black!75, color=white] (1) at (0,2) {\textbf{1}};
	\node[circle, shading=ball, ball color=black!75, color=white] (2) at (1,1) {\textbf{2}};
	\node[circle, shading=ball, ball color=black!75, color=white] (3) at (1,0) {\textbf{3}};
	\node[circle, shading=ball, ball color=black!75, color=white] (4) at (0,-1) {\textbf{4}};
	\node[circle, shading=ball, ball color=black!75, color=white] (5) at (-1,0) {\textbf{5}};
	\node[circle, shading=ball, ball color=black!75, color=white] (6) at (-1,1) {\textbf{6}};
	
	% vertex 1
	\draw[black!75,->,thick](1) to [out=330,in=135,looseness=0](2);
	\draw[black!75,->,thick](1) to [out=315,in=120,looseness=0](3);
	\draw[black!75,->,thick](1) to [out=225,in=45,looseness=0](6);
	% vertex 2
	\draw[black!75,->,thick](2) to [out=240,in=60,looseness=0](4);
	\draw[black!75,->,thick](2) to [out=210,in=30,looseness=0](5);
	% vertex 3
	% vertex 4
	\draw[black!75,->,thick](4) to [out=90,in=270,looseness=0](1);
	\draw[black!75,->,thick](4) to [out=115,in=315,looseness=0](6);
	% vertex 5
	\draw[black!75,->,thick](5) to [out=45,in=255,looseness=0](1);
	\draw[black!75,->,thick](5) to [out=30,in=210,looseness=0](2);
	% vertex 6
	\draw[black!75,->,thick](6) to [out=0,in=180,looseness=0](2);
	\draw[black!75,->,thick](6) to [out=330,in=165,looseness=0](3);
	\draw[black!75,->,thick](6) to [out=270,in=90,looseness=0](5);
\end{tikzpicture}%
}\hfill
&
\resizebox{0.2\textwidth}{!}{% Cycle
\begin{tikzpicture}
	\node[circle, shading=ball, ball color=black!75, color=white] (1) at (0,2) {\textbf{1}};
	\node[circle, shading=ball, ball color=black!75, color=white] (2) at (1,1) {\textbf{2}};
	\node[circle, shading=ball, ball color=black!75, color=white] (3) at (1,0) {\textbf{3}};
	\node[circle, shading=ball, ball color=black!75, color=white] (4) at (0,-1) {\textbf{4}};
	\node[circle, shading=ball, ball color=black!75, color=white] (5) at (-1,0) {\textbf{5}};
	\node[circle, shading=ball, ball color=black!75, color=white] (6) at (-1,1) {\textbf{6}};
	
	% vertex 1
	\draw[black!75,->,thick](1) to [out=330,in=135,looseness=0](2);
	% vertex 2
	\draw[black!75,->,thick](2) to [out=270,in=90,looseness=0](3);
	% vertex 3
	\draw[black!75,->,thick](3) to [out=225,in=45,looseness=0](4);
	% vertex 4
	\draw[black!75,->,thick](4) to [out=135,in=315,looseness=0](5);
	% vertex 5
	\draw[black!75,->,thick](5) to [out=90,in=270,looseness=0](6);
	% vertex 6
	\draw[black!75,->,thick](6) to [out=45,in=225,looseness=0](1);
\end{tikzpicture}%
}\hfill
&
\resizebox{0.2\textwidth}{!}{% Completely Connected
\begin{tikzpicture}
	\node[circle, shading=ball, ball color=black!75, color=white] (1) at (0,2) {\textbf{1}};
	\node[circle, shading=ball, ball color=black!75, color=white] (2) at (1,1) {\textbf{2}};
	\node[circle, shading=ball, ball color=black!75, color=white] (3) at (1,0) {\textbf{3}};
	\node[circle, shading=ball, ball color=black!75, color=white] (4) at (0,-1) {\textbf{4}};
	\node[circle, shading=ball, ball color=black!75, color=white] (5) at (-1,0) {\textbf{5}};
	\node[circle, shading=ball, ball color=black!75, color=white] (6) at (-1,1) {\textbf{6}};
	
	% vertex 1
	\draw[black!75,->,thick](1) to [out=330,in=135,looseness=0](2);
	\draw[black!75,->,thick](1) to [out=295,in=135,looseness=0](3);
	\draw[black!75,->,thick](1) to [out=270,in=90,looseness=0](4);
	\draw[black!75,->,thick](1) to [out=250,in=45,looseness=0](5);
	\draw[black!75,->,thick](1) to [out=225,in=45,looseness=0](6);
	% vertex 2
	\draw[black!75,->,thick](2) to [out=135,in=330,looseness=0](1);
	\draw[black!75,->,thick](2) to [out=270,in=90,looseness=0](3);
	\draw[black!75,->,thick](2) to [out=240,in=60,looseness=0](4);
	\draw[black!75,->,thick](2) to [out=225,in=15,looseness=0](5);
	\draw[black!75,->,thick](2) to [out=180,in=0,looseness=0](6);
	% vertex 3
	\draw[black!75,->,thick](3) to [out=135,in=295,looseness=0](1);
	\draw[black!75,->,thick](3) to [out=90,in=270,looseness=0](2);
	\draw[black!75,->,thick](3) to [out=225,in=45,looseness=0](4);
	\draw[black!75,->,thick](3) to [out=180,in=0,looseness=0](5);
	\draw[black!75,->,thick](3) to [out=165,in=330,looseness=0](6);
	% vertex 4
	\draw[black!75,->,thick](4) to [out=90,in=270,looseness=0](1);
	\draw[black!75,->,thick](4) to [out=60,in=240,looseness=0](2);
	\draw[black!75,->,thick](4) to [out=45,in=225,looseness=0](3);
	\draw[black!75,->,thick](4) to [out=135,in=315,looseness=0](5);
	\draw[black!75,->,thick](4) to [out=115,in=315,looseness=0](6);
	% vertex 5
	\draw[black!75,->,thick](5) to [out=45,in=250,looseness=0](1);
	\draw[black!75,->,thick](5) to [out=15,in=225,looseness=0](2);
	\draw[black!75,->,thick](5) to [out=0,in=180,looseness=0](3);
	\draw[black!75,->,thick](5) to [out=315,in=135,looseness=0](4);
	\draw[black!75,->,thick](5) to [out=90,in=270,looseness=0](6);
	% vertex 6
	\draw[black!75,->,thick](6) to [out=45,in=225,looseness=0](1);
	\draw[black!75,->,thick](6) to [out=0,in=180,looseness=0](2);
	\draw[black!75,->,thick](6) to [out=330,in=165,looseness=0](3);
	\draw[black!75,->,thick](6) to [out=315,in=115,looseness=0](4);
	\draw[black!75,->,thick](6) to [out=270,in=90,looseness=0](5);
\end{tikzpicture}%
}\hfill
&
\resizebox{0.2\textwidth}{!}{% Empty
\begin{tikzpicture}
	\node[circle, shading=ball, ball color=black!75, color=white] (1) at (0,2) {\textbf{1}};
	\node[circle, shading=ball, ball color=black!75, color=white] (2) at (1,1) {\textbf{2}};
	\node[circle, shading=ball, ball color=black!75, color=white] (3) at (1,0) {\textbf{3}};
	\node[circle, shading=ball, ball color=black!75, color=white] (4) at (0,-1) {\textbf{4}};
	\node[circle, shading=ball, ball color=black!75, color=white] (5) at (-1,0) {\textbf{5}};
	\node[circle, shading=ball, ball color=black!75, color=white] (6) at (-1,1) {\textbf{6}};
	
	% vertex 1
	% vertex 2
	% vertex 3
	% vertex 4
	% vertex 5
	% vertex 6
\end{tikzpicture}%
}\\
$\specr = 0.5891$ & $\specr = 0.3000$ & $\specr = 0.2000$ & $\specr = 0.0000$ \\
$\edger = 0.9900$ & $\edger = 0.9939$ & $\edger = 0.0000$ & $\edger = 0.0000$.
\end{tabular}%
}
\end{frame}

\begin{frame}{Sinquefield Cup}
\centering
\resizebox{0.75\textwidth}{!}{% Sinquefield Cup
\begin{tikzpicture}
	\begin{axis}[
		xlabel = Round Number,
		ylabel =  Rankability,
		legend pos = south east,
		cycle list name = black white]
		\addplot+[black] coordinates{
			(1,0.1667)
			(2,0.3333)
			(3,0.4339)
			(4,0.5305)
			(5,0.6555)
			(6,0.7917)
		};
		\addplot+[black!90] coordinates{
			(1,0.1000)
			(2,0.2000)
			(3,0.3000)
			(4,0.4000)
			(5,0.5000)
			(6,0.6000)
			(7,0.7000)
			(8,0.7583)
			(9,0.7832)
			(10,0.7500)
		};
		\addplot+[black!80] coordinates{
			(1,0.1111)
			(2,0.2222)
			(3,0.2873)
			(4,0.3686)
			(5,0.4093)
			(6,0.4578)
			(7,0.5761)
			(8,0.6208)
			(9,0.6711)
		};
		\addplot+[black!70] coordinates{
			(1,0.1111)
			(2,0.1944)
			(3,0.2611)
			(4,0.3325)
			(5,0.4220)
			(6,0.4846)
			(7,0.5152)
			(8,0.5625)
			(9,0.6067)
		};
		\addplot+[black!60] coordinates{
			(1,0.1111)
			(2,0.1944)
			(3,0.2639)
			(4,0.3666)
			(5,0.4328)
			(6,0.4889)
			(7,0.5395)
			(8,0.5719)
			(9,0.6593)
		};
		\addplot+[black!50] coordinates{
			(1,0.1111)
			(2,0.1944)
			(3,0.2662)
			(4,0.3319)
			(5,0.3876)
			(6,0.4882)
			(7,0.5391)
			(8,0.5857)
			(9,0.6389)
		};
		\addplot+[black!40] coordinates{
			(1,0.0909)
			(2,0.1591)
			(3,0.2220)
			(4,0.2826)
			(5,0.3336)
			(6,0.3847)
			(7,0.4252)
			(8,0.4668)
			(9,0.5219)
			(10,0.5572)
			(11,0.6021)
		};
		\legend{2013, 2014, 2015, 2016, 2017, 2018,2019}
	\end{axis}
\end{tikzpicture}%
}
\end{frame}

\begin{frame}{Elo Rating Correlation}
For $i\geq 1$, let $e_{i}$ denote the Elo rating after round $i$.
\vfill
For $i>1$, define the $i$th round Elo rating correlation by 
\[
x_{i}=\textrm{spearmanr}(e_{i},e_{i-1}).
\]
\vfill
We define the years Elo rating correlation by the weighted average
\[
\frac{\sum_{i=1}^{n}ix_{i}}{\sum_{i=1}^{n}i}.
\]
\end{frame}

\begin{frame}{SinquefieldCup Elo Rating Correlation}
\centering
\resizebox{0.75\textwidth}{!}{% Sinquefield Cup
\begin{tabular}{ccc}
Year & Rankability & Elo Rating Correlation \\
\hline
\rowcolor{black!20}2013 & 0.7917 & 0.9600 \\
2014 & 0.7500 & 0.9263 \\
2015 & 0.6711 & 0.8537 \\
2016 & 0.6067 & 0.8205 \\
2017 & 0.6593 & 0.8675 \\
2018 & 0.6389 & 0.9253 \\
\rowcolor{black!20}2019 & 0.6021 & 0.7875 \\
\end{tabular}
}
\vfill
Spearman correlation between Rankability and Elo Rating Correlation is $0.86$.
\end{frame}

\begin{frame}{Big East Elo Rating Correlation}
\centering
\resizebox{0.45\textwidth}{!}{% Big East
\begin{tabular}{ccc}
Year & Rankability & Elo Rating Correlation \\
\hline
1995 & 0.8571 & 0.9253 \\
1996 & 0.8571 & 0.9477 \\
1997 & 0.8149 & 0.8487 \\
1998 & 0.8169 & 0.8835 \\
1999 & 0.8571 & 0.8945 \\
2000 & 0.8571 & 0.9183 \\
2001 & 0.8571 & 0.9397 \\
\rowcolor{black!20}2002 & 0.8571 & 0.9525 \\
2003 & 0.8571 & 0.9075 \\
2004 & 0.6615 & 0.8077 \\
2005 & 0.8375 & 0.9070 \\
2006 & 0.8049 & 0.9218 \\
2007 & 0.6841 & 0.7985 \\
2008 & 0.8049 & 0.8803 \\
2009 & 0.8571 & 0.9392 \\
2010 & 0.7082 & 0.8434 \\
\rowcolor{black!20}2011 & 0.7143 & 0.7312 \\
2012 & 0.7143 & 0.8874 \\
\end{tabular}
}
\vfill
Spearman correlation between Rankability and Elo Rating Correlation is $0.89$.
\end{frame}

\begin{frame}{Mountain West Elo Rating Correlation}
\centering
\resizebox{0.50\textwidth}{!}{% Mountain West
\begin{tabular}{ccc}
Year & Rankability & Elo Rating Correlation \\
\hline
1999 & 0.7078 & 0.8736 \\
\rowcolor{black!20}2000 & 0.8371 & 0.7269 \\
2001 & 0.8571 & 0.9522 \\
2002 & 0.8200 & 0.9192 \\
2003 & 0.7489 & 0.8168 \\
2004 & 0.8377 & 0.8842 \\
2005 & 0.8450 & 0.8441 \\
2006 & 0.8640 & 0.9129 \\
2007 & 0.8750 & 0.9360 \\
2008 & 0.8750 & 0.9538 \\
\rowcolor{black!20}2009 & 1.0000 & 0.9669 \\
2010 & 0.8750 & 0.9320 \\
2011 & 0.8571 & 0.9201 \\
\end{tabular}
}
\vfill
Spearman correlation between Rankability and Elo Rating Correlation is $0.80$.
\end{frame}

\end{document}